\hypertarget{note-de-pruxe9sentation}{%
\section{Note de présentation}\label{note-de-pruxe9sentation}}

\hypertarget{rappel-sur-le-cruxe9dit-renouvelable}{%
\subsection{Rappel sur le crédit
renouvelable}\label{rappel-sur-le-cruxe9dit-renouvelable}}

Un crédit renouvelable est un produit sur le principe assez simple mais
dont le fonctionnement revet certains mécanismes qui le rendent
difficile à appréhender par le consommateur.

Dans un crédit renouvelable, le client dispose d'un droit de tirage sur
une ligne de crédit jusqu'à un encours maximum---nommé montant total du
crédit. En échange il paye des intérets sur le montant effectivement
emprunté.

Durant la vie du contrat, il y a des phases d'augmentation de la
dette---le client se sert soit de l'instrument de paiement lié quand il
existe, soit demande un virement---et des phases d'amortissement---il
paye les intérêts et rembourse une partie du capital emprunté.

Dans certains pays le recours à ce type de crédit est quasi généralisé.
En France il est plutôt utilisé par des clients en situation de
fragilité financière et pour des montants somme-toute relativement
faibles.

L'aspect ouvert de ces contrats amène les prêteurs à proposer des taux
significativement plus élevés que les crédits « fermés » où le client
emprunte une somme fixe d'argent et amortit le capital sur une durée
déterminée à l'avance. En effet les incidents de défaut sont plus
importants.

Les abus directement liés à la formule sont relatifs aux taux élevés et
à la nature ouverte du produit. Ils permettent au prêteur d'accumuler
des intérêts sur une longue durée.

Depuis des années la volonté constante du législateur et du régulateur
tant au niveau européen que national est d'encadrer ces prêts à la
consommation afin d'éviter les abus et de protéger les consommateurs.

Les deux principaux outils mis en place sont la limitation des taux via
la réglementation sur l'usure et la limitation de la durée totale de
remboursement. Conjointement, ils ont pour but d'éviter le piège de la
dette éternelle en forçant les préteurs à limiter la durée maximale et
les taux qu'ils pratiquent.

\hypertarget{rappel-sur-le-cadre-ruxe9glementaire}{%
\subsection{Rappel sur le cadre
réglementaire}\label{rappel-sur-le-cadre-ruxe9glementaire}}

L'immense majorité du dispositif réglementaire sur le crédit en France y
compris immobilier est repris dans le livre 3 du code de la
consommation, articles
\href{https://www.legifrance.gouv.fr/codes/section_lc/LEGITEXT000006069565/LEGISCTA000032221961}{L.311-1
à L.354-6} et
\href{https://www.legifrance.gouv.fr/codes/section_lc/LEGITEXT000006069565/LEGISCTA000032807390}{D.312-1
à R.354-5}. Le code monétaire et financier se contente de quelques
définitions générales et renvoit pour l'essentiel au code de la
consomation.

Pour ce qui est du crédit renouvelable, outre les généralités
applicables à tous les types de crédit (TAEG et usure notamment), les
sections applicables sont les articles
\href{https://www.legifrance.gouv.fr/codes/section_lc/LEGITEXT000006069565/LEGISCTA000032222111}{L.312-57
à L.312-83} et les articles
\href{https://www.legifrance.gouv.fr/codes/section_lc/LEGITEXT000006069565/LEGISCTA000032807458}{D.312-21
à D.312-31}.

\hypertarget{taeg-et-des-seuils-de-lusure.}{%
\subsection{TAEG et des seuils de
l'usure.}\label{taeg-et-des-seuils-de-lusure.}}

Le Taux Annualisé Effectif Global, un outil de comparaison à disposition
du consommateur et du régulateur.

La réglementation---au sens large---impose aux prêteurs de calculer et
de communiquer les taux des offres qu'ils commercialisent pour permettre
aux consommateurs de comparer les offres et de faire un choix éclairé.
Cet outil est aussi à disposition du régulateur en ce qui concerne la
conformité quant à l'usure.

Conceptuellement, il représente le coût moyen de l'argent tout au long
de la durée du crédit.

Son calcul est
\href{https://www.legifrance.gouv.fr/codes/article_lc/LEGIARTI000044536833}{défini
réglementairement} selon les types de contrat de crédit. La méthode
générale est celle de la compensation des flux actuariels qui consiste à
trouver le taux d'inflation qui égalise les décaissements et les
remboursements.

Les taux d'usure sont publiés par trimestre par la Banque de France.
Actuellement, pour ce qui concerne la catégorie des crédits à la
consommation, ils varient selon « le montant du crédit » et le taux va
d'environ 7\% pour les crédits de plus de 6000€ à 20\% pour les crédits
de petit montant, inférieurs à 3000€. Le contrôle du caractère usuraire
se fait sur le TAEG
(\href{https://www.legifrance.gouv.fr/codes/article_lc/LEGIARTI000032226196}{article
L.312-6}).

\hypertarget{cas-des-cruxe9dits-renouvelables}{%
\subsection{Cas des crédits
renouvelables}\label{cas-des-cruxe9dits-renouvelables}}

L'ensemble du marché du crédit renouvelable semble s'être stabilisé sur
des offres proposant un taux débiteur dépendant de l'encours du crédit,
collés aux seuils d'usure, tant sur l'encours que sur les taux. Ils
publient une version annualisée des taux débiteurs en guise de TAEG.

Ces offres « à paliers », semblent en apparence respecter la
réglementation mais sont problématiques à deux égards:

D'une part, une lecture stricte des dispositions sur la formation du
contrat de crédit renouvelable
(\href{https://www.legifrance.gouv.fr/codes/section_lc/LEGITEXT000006069565/LEGISCTA000032222111/\#LEGISCTA000032226066}{articles
L.312-64 à L.312-67}) ainsi que l'article plus général sur tous les
crédits à la consommation
(\href{https://www.legifrance.gouv.fr/codes/article_lc/LEGIARTI000032226196}{article
L.312-6}) laisse planer le doute que les offres à taux débiteurs
variables selon l'encours du crédit soient légales. En effet, il est
fait mention à plusieurs reprises d'un seul taux débiteur, typiquement «
le taux débiteur », et non pas d'une formule plus large comme par
exemple « les dispositions contractuelles régissant les taux débiteurs
». On peut supposer une certaine latitude du régulateur. Une
clarification de la part du législateur serait bienvenue.

D'autre part, l'application de ces paliers conduit à des offres selon
nous usuraires. La réglementation sur l'usure
\href{https://www.legifrance.gouv.fr/codes/article_lc/LEGIARTI000032303335}{(article
L.314-6)} s'assied sur \emph{le}---et non pas les---TAEG du contrat et
le montant du crédit. Dans le cas des crédits renouvelables, le montant
à prendre en compte est le montant total disponible
(\href{https://www.legifrance.gouv.fr/codes/article_lc/LEGIARTI000032226064}{article
L.312-57}) et le calcul précis du TAEG est défini dans la réglementation
nationale qui transpose verbatim les directives européennes ce depuis au
moins 2016. Les règles sont disponibles à
\href{https://www.legifrance.gouv.fr/codes/article_lc/LEGIARTI000044536833}{l'annexe
de l'article R.314-3}.

Ainsi, dans le cas d'un crédit renouvelable le TAEG est obtenu en
simulant un client qui tire immédiatement l'entièreté de la ligne de
crédit (partie II, hypothèse 1°) et rembourse en douze mois par
mensualités égales (partie II, hypothèse 5°). Il est utile de noter que
l'hypothèse 3° de la même partie tient compte des offres qui
appliqueraient un amortissement séparé pour chaque utilisation.

Nous avons pû constater que les documentations disponibles ainsi que les
documents précontractuels mentionnent rarement sinon jamais le TAEG
conventionnel des contrats alors qu'il s'agit d'une information
normalement obligatoire. L'argument que le caractère dynamique du crédit
empêche un calcul de TAEG est irrecevable puisque ce cas a été addressé
par la réglementation.

Appliqué aux offres disponibles sur le marché en prenant un crédit d'un
montant total un peu au delà du plafond de 6000€, on arrive à des offres
dont le TAEG conventionnel dépasse presque du double le taux de l'usure.

Nous sommes donc en présence d'une faillite multiple: les consommateurs
sont privés d'une information normalement obligatoire lors du
précontrat; sont face à des offres de fait usuraires, insincères et de
légalité douteuse de part leur construction; et enfin le régulateur est
privé d'un moyen de contrôle car il doit calculer lui-même le TAEG et ne
bénéficie pas des alertes consommateurs.

\hypertarget{proposition-dun-taeg-dynamique-en-cours-de-contrat}{%
\subsection{Proposition d'un TAEG dynamique en cours de
contrat}\label{proposition-dun-taeg-dynamique-en-cours-de-contrat}}

Le TAEG conventionnel est utile en situation pré-contractuelle,
cependant il repose sur un scénario d'utilisation qui ne peut pas
refléter la diversité des usages offerts par les offres de type crédit
renouvelable.

Il nous semble important de fournir lors de l'exécution du contrat un
moyen au client de contrôler le coût de son crédit notamment en phase
d'amortissement et de permettre au régulateur de s'appuyer dessus pour
le contrôle de la conformité d'un contrat relativement aux règles sur
l'usure.

À cette fin, nous proposons d'introduire la notion de TAEG dynamique qui
reprendra les paramètres de la phase d'amortissement, et devra être
communiqué par le prêteur lors du relevé périodique.

Nous proposons de réutiliser la formule d'équivalence des flux avec des
paramètres adaptés. Le point de départ du calcul est fixé au moment de
la dernière utilisation active du crédit. Il consiste à équilibrer d'une
part le capital effectivement emprunté suite à cette dernière
utilisation et d'autre part les remboursements déjà effectués ainsi que
les remboursements prévisibles selon les termes du contrat jusqu'à
l'extinction de la dette, en prenant l'hypothèse d'une absence
d'utilisation active ultérieure du crédit.

Il est à noter qu'une utilisation ultérieure du crédit renouvelable ou
un remboursement anticipé futur invalidera les hypothèses de calcul de
ce TAEG dynamique---mais pas forcément sa valeur selon la manière dont
le prêteur a structuré son offre. Cela ne nous semble pas problématique
dans le sens qu'un emprunt supplémentaire correspond à une nouvelle
prise de risque de la part du prêteur---justifiant une invalidation de
toutes les hypothèses---et toute action manuelle altère de toutes façons
les paramètres de la phase d'amortissement à savoir le coût total du
crédit, le nombre de mensualités restantes, et leur montant. Le choix du
point de départ au point de la dernière utilisation permet, si le client
s'en tient aux plan d'amortissement prévu, que la valeur calculée reste
constante jusqu'à extinction de la dette. La réglementation limitant le
nombre de mensualités (articles
\href{https://www.legifrance.gouv.fr/codes/article_lc/LEGIARTI000032226041}{L.312-65}
et
\href{https://www.legifrance.gouv.fr/codes/article_lc/LEGIARTI000032807472}{D.312-27}
) a fait ce même choix technique. Une présentation simple lors du relevé
périodique permet de lever toute ambiguité, par exemple « Si vous
n'utilisez plus votre crédit, le solde restant sera réglé en X
mensualités d'un montant de Y EUR à un TAEG de Z\% pour un montant total
de T EUR ».

À noter les points suivants :

\begin{itemize}
\tightlist
\item
  la double définition d'un TAEG ex-ante et ex-post n'est pas un concept
  inédit: il est déjà implémenté dans le cas des découverts en compte,
  type de crédits qui sont sujets à une utilisation dynamique, il n'y a
  donc pas innovation réglementaire.
\item
  Pour ce qui est de la conformité européenne, il ne devrait pas y avoir
  d'incidence: d'une part la définition du TAEG en cours de contrat est
  égale au TAEG conventionnel dans le cas de l'usage du contrat de
  crédit selon la formule conventionnelle; ce TAEG dynamique ne se
  subsitute pas au TAEG conventionnel pré-contrat, par construction; et
  enfin son usage potentiel pour contrôle de l'usure reste une
  compétence nationale.
\item
  Cette définition protège les consommateurs et le régulateur d'une
  application malicieuse par le prêteur de paramètres de taux débiteurs
  qui optimiserait le TAEG conventionnel tout en cachant une offre
  insincère dans la pratique car il n'y a plus aucune variable
  d'interprétation quant à la durée et aux montants.
\item
  La manière la plus simple pour les prêteurs de rendre leurs offres
  lisibles et de s'assurer une conformité avec l'esprit de la
  réglementation sur l'usure est de geler le taux débiteur au moment de
  la dernière utilisation du crédit, une utilisation ultérieure pouvant
  déclencher un réajustement du taux. Némmoins ce n'est pas ni à la loi
  ni à la réglementation de dicter au prêteur comment structurer son
  offre. Le TAEG pré-contractuel et le TAEG dynamique fournissent une
  analyse « boîte noire » des coûts du prêt sans devoir rentrer dans les
  détails du fonctionnement du contrat.
\item
  La réglementation actuelle utilise la dernière utilisation du crédit
  comme point de départ pour certains paramètres dont notamment la durée
  de remboursement et la mensualité minimale (L.312-65 et D.312-27), il
  n'y a donc pas introduction d'un « jalon » nouveau lors de l'exécution
  du contrat de prêt.
\item
  Les prêteurs disposent déjà des éléments nécéssaires au calcul car
  l'essentiel est disponible parmis les mentions obligatoires sur le
  relevé de période. Par ailleurs les prêteurs disposent des logiciels
  de calcul de flux actuariels de part leur métier et des autres
  exigences réglementaires qui s'appliquent aux contrats de prêt.
\item
  Nous considérons que les dispositifs tant législatifs que
  réglementaires sur l'usure donnent suffisamment de moyens au
  régulateur, il ne nous semble pas opportun à ce stade de toucher à
  cette section du code de la consommation. Nous ne prétendons pas nous
  substituer au régulateur dans son appréciation du marché ou de sa
  décision de réglementation de l'usure, nous nous contentons de lui
  donner un outil supplémentaire pour orienter son action.
\end{itemize}

\hypertarget{moyens-daction}{%
\subsection{Moyens d'action}\label{moyens-daction}}

\hypertarget{ruxe9gulateur-acpr-associations-de-consommateurs}{%
\subsubsection{Régulateur / ACPR / Associations de
consommateurs}\label{ruxe9gulateur-acpr-associations-de-consommateurs}}

La persistence d'offres sur le marché qui sont de fait usuraires nous
semble refléter l'existence d'un point aveugle du régulateur. Il serait
bon d'être vigilant sur cet aspect.

\hypertarget{cadre-luxe9gislatif}{%
\subsubsection{Cadre législatif}\label{cadre-luxe9gislatif}}

{[}Sénat, commission des affaires économiques{]}

Nous proposons de:

\begin{itemize}
\item rajouter au premier alinéa de l'article
\href{https://www.legifrance.gouv.fr/codes/section_lc/LEGITEXT000006069565/LEGISCTA000032222141/?anchor=LEGIARTI000035731394}{L.312-71}
du code de la consommation relatif au relevé périodique un 7°-bis qui
impose au prêteur de fournir le TAEG dynamique. 
\item supprimer la mention au taux effectif global au 4°. En effet il s'agit
simplement d'une annualisation du taux débiteur de période, qui n'a pas
grand intérêt pour un consommateur dans le cas d'une offre à taux débiteurs
dynamiques. Néammoins dans le cas d'une formule à taux débiteur unique ou
fixé au moment de la dernière utilisation du crédit, le TAEG dynamique sera
bien égal à l'annualisation de ce taux débiteur. Cette suppression n'est
donc pas dommageable sur les formules simples et apporte de la clareté pour
les formules plus complexes.
\end{itemize}

Il faut aussi noter que la partie législative dans toute cette section
du code parle de Taux Effectif Global, notion désuète remplacée depuis
par le Taux Annuel Effectif Globlal, une correction serait utile. Il est
proposé ici de conserver taux effectif global par cohérence de
l'ensemble du texte.

Cet amendement a pour objet d'ancrer le concept de TAEG dynamique et de
permettre un point de comparaison important pour les clients: combien
cela me coûte de payer cette dette. Si le taux paraît excessif au client
il peut chercher à racheter le crédit pour un crédit amortissable, ou
mobiliser d'autres moyens de financement moins chers.

\begin{quote}
Au premier alinéa de l'article L.312-71 du code de la consommation:

\begin{itemize}
\item
  Au 4° les mots « et le taux effectif global » sont supprimés.
\item
  Après le 7°, il est inséré un 7°-bis ainsi rédigé : « 7°-bis Le taux
  effectif global représentant l'opération d'extinction de la dette dans
  le cas où il n'y a plus d'utilisation active ultérieure. Les modalités
  de calcul sont déterminées par un décret en Conseil d'État; »
\end{itemize}
\end{quote}

(On pourra vouloir créer un nouvel article au code de la consommation
qui consacre le double dispositif du TAEG et y faire ensuite référence
depuis l'article L.312-71. Il est inutile à ce stade de discuter de ce
point de légistique)

\hypertarget{cadre-ruxe9glementaire}{%
\subsubsection{Cadre réglementaire}\label{cadre-ruxe9glementaire}}

Travailler avec l'ACPR pour la rédaction du mode de calcul de ce TAEG
d'extinction ou dynamique.

(L'article L.312-71 n'ayant pas d'équivalent dans la partie
réglementaire, on voudra possiblement créer un R.312-71 de renvoi vers
la section qui dispose du calcul de TAEG. Il n'est pas utile de discuter
de ce point de légistique à ce stade.)

Nous proposons la création d'un article R. 314-7-1, inséré après
l'article R.314-7 qui concerne les découverts en compte:

\begin{quote}
Art. R.314-7-1. -- Pour un crédit renouvelable, lorsque le taux annuel
effectif global est calculé avant toute utilisation, le calcul est
effectué selon les modalités prévues à l'annexe du R.314-3

Pendant l'utilisation du crédit, le taux annuel effectif global est
calculé en appliquant la méthode des flux équivalents en rapportant :

1° D'une part, le capital effectivement emprunté au terme de la période
correspondant à la dernière utilisation active du crédit;

2° D'autre part, les remboursements déjà effectués depuis cette période
ainsi que les remboursements futurs prévus selon les modalités du
contrat, jusqu'à extinction complète du capital emprunté, en supposant
l'absence de nouvelles utilisations.

Le taux ainsi déterminé est communiqué à l'emprunteur dans les relevés
périodiques prévus à l'article L.312-71.
\end{quote}

\hypertarget{mise-en-pratique}{%
\section{Mise en pratique}\label{mise-en-pratique}}

\hypertarget{introduction}{%
\subsection{Introduction}\label{introduction}}

Afin de rendre compréhensible le mécanisme qui conduit à des offres
usuraires, nous proposons de construire et comparer deux formules
courantes: le crédit amortissable et le crédit renouvelable « à paliers
» tels qu'un consommateur peut les trouver sur le marché français du
crédit à la consommation.

Les taux débiteurs seront communs aux deux offres et sont fixés au
regard des seuils d'usure du 4è trimestre 2025.

Le montant emprunté est choisi selon un scenario réaliste et de manière
à balayer l'ensemble du spectre des taux d'usure.

Nous étudierons trois scénarios d'utilisation:

\begin{itemize}
\tightlist
\item
  crédit amortissable de 6 500€ à taux fixe remboursable sur trois ans à
  mensualités fixes
\item
  crédit in fine de 6 500€ à taux fixe sur trois ans
\item
  crédit renouvelable « à paliers » de 6 500€ remboursé sur trois ans à
  mensualités fixes
\end{itemize}

Nous discuterons pour chaque scénario du prix total pour le consommateur
ainsi que le TAEG obtenu. Un tableau récapitulatif sera fourni en fin de
document.

Nous montrerons enfin comment un prêteur peut, en adaptant une offre à
paliers, respecter l'esprit des réglementations sur l'usure.

\hypertarget{construction-des-offres}{%
\subsection{Construction des offres}\label{construction-des-offres}}

\hypertarget{seuils-dusure}{%
\subsubsection{Seuils d'usure}\label{seuils-dusure}}

Contrairement aux crédits immobiliers, les seuls d'usure pour les
crédits à la consommation---appelés« crédits de trésorerie » sont
déterminés en fonction du montant du crédit.

Ils sont publiés par la banque de France. Pour le 4è trimestre 2025 ils
sont disponibles sur la page
\href{https://www.banque-france.fr/fr/statistiques/taux-et-cours/taux-dusure-2025-q4}{Taux
d'usure --- 2025-Q4} et sont rappelés dans le tableau suivant:

\begin{center}
\begin{tabular}{lcc}
\hline
\textbf{Catégorie} & \textbf{Taux moyen T3-2025} & \textbf{Taux d'usure pour T4-2025} \\
\hline
m $\leqslant$ 3000€ & 17,62\% & 23,49\% \\
3000€ $<$ m $\leqslant$ 6000€ & 11,78\% & 15,71\% \\
6000€ $<$ m & 6,55\% & 8,73\% \\
\hline
\end{tabular}
\end{center}

Ces taux sont à comparer au TAEG (Taux Annualisé Effectif Global) du
prêt considéré.

\hypertarget{taux-duxe9biteurs}{%
\subsubsection{Taux débiteurs}\label{taux-duxe9biteurs}}

\hypertarget{rappel-sur-le-taux-duxe9biteur-et-le-taux-annualisuxe9}{%
\paragraph{Rappel sur le taux débiteur et le taux
annualisé}\label{rappel-sur-le-taux-duxe9biteur-et-le-taux-annualisuxe9}}

Les crédits sont paramétrés en fonction du taux débiteur. Pour obtenir
le taux de période, qui est le coefficient multiplicateur appliqué au
capital après chaque période il est d'usage de diviser le taux par le
nombre de périodes dans une année.

Ex. un taux débiteur de 7,5\% avec une périodicité de un mois aura un
coefficient multiplicateur de période de 1 + 7,5/1200 = 1,00625 soit un
taux de période de 6,25‰ (pour-mille).

Contrairement à un taux débiteur un taux annualisé tient compte des
intérêts composés sur un an, et donc la relation entre un taux débiteur
et le taux annualisé s'obtient par la formule suivante: (1 + t / p)\^{}p
- 1, t étant le taux et p étant le nombre de périodes dans l'année. Ce
taux annualisé correspond à l'augmentation du montant emprunté dans le
cas où le client ne rembourserait rien pendant un an, laissant les
intérêts capitaliser.

Sur l'exemple du prêt à 7,5\% avec une période de un mois, le taux
annualisé sort à (1 + 7,5 / 1200)\^{}12 - 1 = 7,76\%.

Dans le cas des crédits de durée relativement faible cette différence
n'est pas significative pour comparer les coûts d'un crédit. C'est le
cas des crédits à la consommation.

Le point le plus important à retenir est que ce taux annualisé n'est pas
forcément égal au TAEG qui considère l'ensemble de la vie du crédit et
en incluant les frais prévisibles, il faut voir le TAEG comme le taux
moyen sur toute la vie du prêt.

\hypertarget{taux-duxe9biteurs-appliquuxe9s}{%
\paragraph{Taux débiteurs
appliqués}\label{taux-duxe9biteurs-appliquuxe9s}}

Pour fixer les idées nous proposons de construire l'exemple avec les
taux débiteurs suivants, qui sont choisis pour être environ 10\% en
dessous du seuil d'usure :

%\begin{longtable}[]{@{}lll@{}}
\begin{longtable}{lll}
\toprule
\textbf{Montant emprunté} & \textbf{Taux débiteur} & \textbf{Taux annualisé} \\
\midrule
\endhead
m $\leqslant$ 3000€ & 19,33\% & 21,14\% \\
3000€~$<$ m $\leqslant$ 6000€ & 13,30\% & 14,14\% \\
6000€~$<$ m & 7,59\% & 7,86\% \\
\bottomrule
\end{longtable}

\hypertarget{scuxe9narios-dusage}{%
\subsection{Scénarios d'usage}\label{scuxe9narios-dusage}}

Le montant emprunté est de 6500€ et la durée de vie du prêt est de trois
ans, soit trente-six mois.

Rappel qu'un crédit fonctionne de la manière suivante. À la fin de
chaque période le capital est:

\begin{enumerate}
\def\labelenumi{\arabic{enumi}.}
\tightlist
\item
  augmenté des intérêts
\item
  diminué du remboursement
\end{enumerate}

Le paiement de la cotisation d'assurance est en sus.

Quand le remboursement est inférieur aux intérêts, l'amortissement est
dit négatif et le capital emprunté s'accroit; quand il est égal
l'amortissement est nul; et finalement quand il est supérieur
l'amortissement est positif.

En cas d'événement intervenant en cours de période, des intérêts dits «
intercalaires » sont facturés. Pour simplifier nous n'allons pas
considérer cet événement, cela ne change les coûts que à la marge.

\hypertarget{cruxe9dit-amortissable-uxe0-mensualituxe9s-constantes}{%
\subsubsection{Crédit amortissable à mensualités
constantes}\label{cruxe9dit-amortissable-uxe0-mensualituxe9s-constantes}}

C'est la forme la plus courante du prêt, même si ce n'est pas la plus
simple à comprendre. Comme la mensualité est constante, au début cette
mensualité rembourse beaucoup d'intérêts---pusique le capital sur
lesquels ils s'appliquent est important---et moins de capital---pusique
que la mensualité est amputée des intérêts. À la fin les intérêts sont
moindres puisque que le capital restant est faible et donc la mensualité
rembourse plus de capital.

Pour se faire une idée approximative du coût d'un prêt de ce type, on
peut considérer en première approximation que le capital « en moyenne »
sur la durée du prêt va être d'environ la moitié du capital emprunté, et
que les intérêts vont s'appliquer sur cette moitié de capital pour toute
la durée du prêt.

La mensualité est calculable exactement, et fait intervenir la
\href{https://fr.wikipedia.org/wiki/Mensualit\%C3\%A9}{somme d'une série
géométrique}. Voir aussi cette page liée:
\href{https://images-archive.math.cnrs.fr/Emprunts-mensualites-interet-taux-TEG-risque-de-taux.html?lang=fr}{Emprunts
: mensualités, intérêt, taux, TEG, risque de taux}

Le capital emprunté K est de 6500€. Le taux de période est donc
7,59\%/12 = 6,325‰

Ce capital se rembourse en 35 mensualités de 202,46€, et une dernière
mensualité de 202.41€. Le coût total du crédit est de 7288,51€. Le TAEG
est égal au taux débiteur annualisé, ici 7,86\%.

\hypertarget{cruxe9dit-in-fine}{%
\subsubsection{Crédit in fine}\label{cruxe9dit-in-fine}}

Le crédit in fine consiste à rembourser tout le capital en une fois à la
fin et ne payer que les intérêts à chaque fin de période.

Tel quel ce scénario est assez rare dans le cadre du crédit à la
consommation mais il est inclus ici car il maximise le coût d'un prêt
pour un certain taux. Pour les crédits auto il est courant de pratiquer
un crédit intermédiaire où une partie du capital est remboursé chaque
mois et le ballon correspondant à la valeur résiduelle du véhicule est
soldé à la fin, soit en rendant le véhicule, soit en payant le ballon:
c'est la formule de Location avec Option d'Achat.

Le coût total est typiquement le double du coût d'un crédit amortissable
car les intérêts s'appliquent sur la totalité du capital pendant toute
la durée du prêt, contrairement au crédit amortissable sur lesquels les
intérêts s'appliquent sur une portion qui diminue au fil du temps.

Ce cas est très facile à calculer car la mensualité est égale aux
intérêts.

Le capital emprunté K est de 6500€. Le taux de période est donc
7,59\%/12 = 6,325‰

Dans notre cas les intérêts reviennent à 6500 x 6,325‰ = 41,11€ ce qui
donne la mensualité.

Ce capital se rembourse en 35 mensualités de 41,11€, et une dernière de
6541,11€. Le coût total du crédit est de 7979,96€. Le TAEG est égal au
taux débiteur annualisé, ici 7,86\%.

\hypertarget{cruxe9dit-renouvelable-uxe0-paliers}{%
\subsubsection{Crédit renouvelable à
paliers}\label{cruxe9dit-renouvelable-uxe0-paliers}}

Comme discuté dans la note principale, le prêteurs font varier le taux
débiteur selon l'encours sous le prétexte que le crédit est
renouvelable. Cela va conduire mécaniquement à un coût plus élevé des
intérêts en fin de prêt, quand le solde passe en dessous des seuils qui
conduisent à des taux élevés.

Trouver la mensualité constante n'est pas évident, il n'y a pas de
relation mathématique qui conduit au montant. Il faut procéder par
approximation successives.

Le capital emprunté est de 6500€

On trouve la mensualité constante 220,06€ par la méthode de la sécante.

Ce capital se rembourse en 35 mensualités de 220,06€ et une dernière de
218,43€. Le coût total du crédit est de 7920,53€. Le TAEG pour cette
opération est calculé à 14.17\%.

Le TAEG conventionnel est obtenu en remboursant sur 12 mois l'entièreté
de la ligne de crédit. Dans ce cas le capital est remboursé avec 11
mensualités de 582,16€ et une dernière de 582,07€. Le coût total du
crédit est de 6985,83€. Le TAEG pour cette opération est calculé à
14,39\%. Comme on peut le constater la différence de TAEG n'est pas
significative.

Afin que le lecteur se rende bien compte du fonctionnement du prêt, le
tableau d'amortissement complet est reproduit
\protect\hyperlink{tableau-damortissement}{en annexe}. Il convient de
prêter attention aux intérêts facturés à partir des échéances n°4 et
n°22: juste après l'échéance n°3, le capital emprunté passe en dessous
de 6500€, et de la même façon juste après l'échéance n°21 le capital
emprunté passe en dessous de 3000€. Dans les deux cas les intérêts
facturés subissent un saut et l'amortissement diminue d'autant car le
taux débiteur est fonction de l'encours. La mensualité constante « cache
» ce fonctionnement.

Dans un crédit classique, on s'attend à ce que la proportion des
intérêts payés par la mensualité diminue avec le temps---on parle de
comportement monotone. Ce n'est absolument pas le cas ici.

\hypertarget{tableau-comparatif}{%
\subsection{Tableau comparatif}\label{tableau-comparatif}}

\begin{center}
\resizebox{\textwidth}{!}{
  \begin{tabular}{lccccl}
  \hline
  \textbf{Scenario} & \textbf{Mensualités} & \textbf{Coût total du prêt} & \textbf{Intérêts} & \textbf{TAEG final} & \textbf{Note} \\
  \hline
  Crédit amortissable & 202,46€ x 36 & 7288,51€ & 788,51€ & 7,86\% & Crédit le moins cher \\
  Crédit in fine & 41,11€ x 35 + 6541,11€ & 7979,96€ & 1479,96€ & 7,86\% & Crédit le plus cher \\
  Crédit renouvelable à paliers & 220,06€ x 36 & 7920,53€ & 1420,53€ & 14,17\% & Offre usuraire \\
  \hline
  \end{tabular}
}
\end{center}

On notera que l'amortissement du crédit renouvelable coûte quasiment
aussi cher qu'un crédit in fine, et de fait rien n'interdit à
l'emprunteur de s'en servir de cette façon en réutilisant le crédit de
manière à rester dans la tranche à taux la plus faible. Du point de vue
du prêteur ce comportement est à risque et devrait être pénalisé par une
charge de la dette plus élevée. Un client qui éteint sa dette au fur et
à mesure diminue le risque de défaut. Il est anormal d'obtenir un coût
quasi identique en faisant fonctionner le prêt dans cette configuration.

On notera enfin qu'un même TAEG peut cacher des disparités importantes
de charge de la dette selon la formule d'amortissement choisie, même
pour une durée égale.

\hypertarget{conclusion}{%
\subsection{Conclusion}\label{conclusion}}

Appliquer un taux débiteur selon l'encours conduit à des offres
\emph{grossièrement} usuraires. Ce type d'offre est selon nous
insincère, induit le consommateur en erreur sur les coûts réels et ne
respecte pas ni l'esprit ni la lettre de la réglementation sur l'usure.

Une manière simple pour un prêteur de construire une offre conforme à
l'usure, selon le TAEG conventionnel et selon les usages raisonnables,
consiste à geler le taux débiteur au moment de la dernière utilisation.
La phase d'amortissement qui suit devient alors celle d'un crédit à taux
fixe et mensualités constantes.

Cette manière de procéder est cohérente vis-à-vis du risque, car un
crédit en fin d'amortissement est moins risqué qu'un crédit en usage
actif. De même si vers la fin de l'amortissement le client réutilise sa
ligne de crédit, il est alors légitime qu'une nouvelle phase
d'amortissement s'ouvre avec un taux plus élevé car cette situation
constitue une nouvelle prise de risque pour le prêteur.

Bien sûr cette approche n'interdit pas au prêteur de réviser son abaque
de taux, conformément à l'article
\href{https://www.legifrance.gouv.fr/codes/article_lc/LEGIARTI000035731391}{L.312-72
du code de la consommation}. Il convient évidemment dans cette hypothèse
de vérifier que cela ne provoque pas un comportement usuraire. Un
amortissement démarré dans une des tranches des tableaux d'usure doit y
rester, tant que le client ne réutilise pas sa facilité de crédit.

\hypertarget{annexe}{%
\subsection{Annexe}\label{annexe}}

\hypertarget{tableau-damortissement}{%
\subsubsection{Tableau d'amortissement}\label{tableau-damortissement}}

Tableau d'amortissement du crédit renouvelable sur 36 mois pour 6500€
empruntés et avec mensualité constante

\begin{center}
\resizebox{\textwidth}{!}{
  \begin{tabular}{crrrrrl}
  \hline
  \textbf{mois} & \textbf{décaissement} & \textbf{mensualité} & \textbf{intérêts} & \textbf{amortissement} & \textbf{capital restant} & \textbf{note} \\
  \hline
  0 & 6500,00 & 0,00 & 0,00 & 0,00 & 6500,00 & \\
  1 & 0,00 & 220,06 & 41,11 & 178,95 & 6321,05 & \\
  2 & 0,00 & 220,06 & 39,98 & 180,08 & 6140,97 & \\
  3 & 0,00 & 220,06 & 38,84 & 181,22 & 5959,75 & \\
  4 & 0,00 & 220,06 & 66,05 & 154,01 & 5805,74 & nv. tx. 13,30\% \\
  5 & 0,00 & 220,06 & 64,35 & 155,71 & 5650,03 & \\
  6 & 0,00 & 220,06 & 62,62 & 157,44 & 5492,59 & \\
  7 & 0,00 & 220,06 & 60,88 & 159,18 & 5333,41 & \\
  8 & 0,00 & 220,06 & 59,11 & 160,95 & 5172,46 & \\
  9 & 0,00 & 220,06 & 57,33 & 162,73 & 5009,73 & \\
  10 & 0,00 & 220,06 & 55,52 & 164,54 & 4845,19 & \\
  11 & 0,00 & 220,06 & 53,70 & 166,36 & 4678,83 & \\
  12 & 0,00 & 220,06 & 51,86 & 168,20 & 4510,63 & \\
  13 & 0,00 & 220,06 & 49,99 & 170,07 & 4340,56 & \\
  14 & 0,00 & 220,06 & 48,11 & 171,95 & 4168,61 & \\
  15 & 0,00 & 220,06 & 46,20 & 173,86 & 3994,75 & \\
  16 & 0,00 & 220,06 & 44,28 & 175,78 & 3818,97 & \\
  17 & 0,00 & 220,06 & 42,33 & 177,73 & 3641,24 & \\
  18 & 0,00 & 220,06 & 40,36 & 179,70 & 3461,54 & \\
  19 & 0,00 & 220,06 & 38,37 & 181,69 & 3279,85 & \\
  20 & 0,00 & 220,06 & 36,35 & 183,71 & 3096,14 & \\
  21 & 0,00 & 220,06 & 34,32 & 185,74 & 2910,40 & \\
  22 & 0,00 & 220,06 & 46,88 & 173,18 & 2737,22 & nv. tx: 19,33\% \\
  23 & 0,00 & 220,06 & 44,09 & 175,97 & 2561,25 & \\
  24 & 0,00 & 220,06 & 41,26 & 178,80 & 2382,45 & \\
  25 & 0,00 & 220,06 & 38,38 & 181,68 & 2200,77 & \\
  26 & 0,00 & 220,06 & 35,45 & 184,61 & 2016,16 & \\
  27 & 0,00 & 220,06 & 32,48 & 187,58 & 1828,58 & \\
  28 & 0,00 & 220,06 & 29,46 & 190,60 & 1637,98 & \\
  29 & 0,00 & 220,06 & 26,39 & 193,67 & 1444,31 & \\
  30 & 0,00 & 220,06 & 23,27 & 196,79 & 1247,52 & \\
  31 & 0,00 & 220,06 & 20,10 & 199,96 & 1047,56 & \\
  32 & 0,00 & 220,06 & 16,87 & 203,19 & 844,37 & \\
  33 & 0,00 & 220,06 & 13,60 & 206,46 & 637,91 & \\
  34 & 0,00 & 220,06 & 10,28 & 209,78 & 428,13 & \\
  35 & 0,00 & 220,06 & 6,90 & 213,16 & 214,97 & \\
  36 & 0,00 & 218,43 & 3,46 & 214,97 & 0,00 & \\
  \hline
  \end{tabular}
}
\end{center}

Ce tableau illustre que le coût du crédit ne suit pas une trajectoire
régulière : la proportion d'intérêts dans la mensualité peut augmenter
en cours de remboursement, phénomène incompatible avec la logique d'un
amortissement classique.

\hypertarget{exemples-doffres}{%
\section{Exemples d'offres}\label{exemples-doffres}}

Nous avons récolté quelques offres non conformes. Elles sont présentées
ici de manière anonymisées.

\hypertarget{offre-n1}{%
\subsection{Offre n°1}\label{offre-n1}}

Cet établissement de crédit connu propose un crédit renouvelable « par
palier » de 6500€ à rembourser en 54 échéances de 163€ et une dernière
de 81,55€, pour un montant total de 8883,55€ Le TAEG n'est pas
communiqué, seuls sont communiqués les taux débiteurs par paliers et
leur annualisation en guise de TAEG. Cette offre n'est pas conforme à la
réglementation sur les éléments obligatoirements communiqués à
l'emprunteur, et est de plus usuraire. Le TAEG de l'échelonnement
proposé est à 15,33\%.

Cette offre ne devrait plus être commercialisée selon nous.

\begin{center}
\includegraphics[scale=0.3]{../docs/img/1_1_caracteristiques.png}\\
\includegraphics[scale=0.3]{../docs/img/1_2_cout_du_credit.png}\\
\includegraphics[scale=0.3]{../docs/img/1_3_cout_du_credit.png}\\
\end{center}

\hypertarget{offre-n2}{%
\subsection{Offre n°2}\label{offre-n2}}

Cet établissement de crédit connu propose un tableau de mensualités et
de taux débiteurs. Sans information supplémentaire il est difficile de
juger du caractère usuraire ou non de l'offre. L'exemple donné n'est pas
suffisant au regard de la réglementation qui requiert des exemples de
500€, 1500€ et 3000€ dans les documentations commerciales selon
l'article
\href{https://www.legifrance.gouv.fr/codes/article_lc/LEGIARTI000032807460}{D.312-21
du code de la consommation}.

En laissant les mensualités et les taux varier selon l'encours comme le
laisse supposer les intitulés du tableau, un capital emprunté de 6500€
en option « lente » s'amortit en 97 mensualités (variables), pour un
coût total 8958,28€ et le TAEG ressort à 13.81\%.

Une demande de précisions par l'ACPR auprès de cet établissement de
crédit serait justifiée.

\begin{center}
\includegraphics[scale=0.4]{../docs/img/2_1_cout_du_credit.jpeg}\\
\includegraphics[scale=0.4]{../docs/img/2_2_exemple.png}
\end{center}

\hypertarget{offre-n3}{%
\subsection{Offre n°3}\label{offre-n3}}

Cet établissement de crédit connu propose un crédit renouvelable de
3500€ remboursable en 45 mensualités de 108,50€ et une dernière de
17,60€, pour un coût total de 4900,10€. La présentation du TAEG n'est
pas conforme.

Le TAEG de l'amortissement proposé ressort à 20,37\%.

Cette offre ne devrait plus être commercialisée selon nous.

\begin{center}
\includegraphics[scale=0.3]{../docs/img/3_1_proposition_commerciale.jpeg}
\end{center}
